%%%%%%%%%%%%%%%%%%%%%%%%%%%%%%%%%%%%%%%%%%%%%%
% THIS FILE CONTAINS DEFINITIONS AND COMMANDS
%%%%%%%%%%%%%%%%%%%%%%%%%%%%%%%%%%%%%%%%%%%%%%

%% Import useful packages.
%\usepackage{times}
%\usepackage{amsmath}                           	% for 'case'
%\usepackage{amsthm}                           	% for theorems and proofs
%\usepackage{mathrsfs}                          	% script math font
%\usepackage{url}                               	% allow use of URLs
%\usepackage{latexsym}                           % some math symbols, like \Box
%\usepackage{pseudocode}                         % for pseudocode environment
%\usepackage{subfigure}                           % allow use of subfigures

%% figure path
%\newif\ifpdf \ifx\pdfoutput\undefined
%\pdffalse
%\usepackage[dvips]{graphicx}
%\DeclareGraphicsExtensions{.eps,.epsi}
%\else
%\pdfoutput=1
%\pdfcompresslevel=9
%\pdftrue
%\usepackage[pdftex]{graphicx}
%\DeclareGraphicsExtensions{.pdf}
%\fi
%\graphicspath{{figures/}}

%% set up theorem block definitions
%\newtheorem{theorem}{Theorem}[section]
%\newtheorem{corollary}[theorem]{Corollary}
%\renewcommand{\thepseudonum}{\roman{pseudonum}}

%% quotations at beginning of chapters
%\newcommand{\myquote}[3]{
%\begin{flushright}
%\begin{minipage}[t]{#1}
%\itshape
%#3 \\
%\upshape
%\end{minipage}
%\\[-1mm]
%#2
%\end{flushright}
%\vspace{5mm}
%}


% Define some useful commands we will use repeatedly.
%\newcommand{\bfm}[1]{{\bf #1}}              % proper boldface for math environment
\newcommand{\bfm}[1]{{\mbox{\boldmath{$#1$}}}}               % proper boldface for math environment
%\newcommand{\bfm}[1]{{\boldsymbol{#1}}}               % proper boldface for math environment

\newcommand{\op}[1]{\mathcal{#1}}
%\newcommand{\vektor}[1]{{\bf #1}}
%\renewcommand{\vec}[1]{\bfm{#1}} % italicized boldface
\renewcommand{\vec}[1]{{\bf #1}}
\newcommand{\q}{\vec{q}} % coordinates
\newcommand{\p}{\vec{p}} % momenta
\newcommand{\z}{\vec{z}} % phase space point
\newcommand{\x}{\vec{x}} % general vector x
\newcommand{\M}{\bfm{M}} % diagonal mass matrix
\newcommand{\grad}{\nabla}
\newcommand{\timeavg}[1]{\overline{#1}}                    % time average over a trajectory
\newcommand{\T}{\mathrm{T}}                                % T used in matrix transpose
\newcommand{\tauarrow}{\stackrel{\tau}{\rightarrow}}       % the symbol tau over a right arrow
\newcommand{\expect}[1]{\langle #1 \rangle}                % <.> for denoting expectations over realizations of an experiment or thermal averages
\newcommand{\estimator}[1]{\hat{#1}}                       % estimator for some quantity from a finite dataset.
\newcommand{\code}[1]{{\tt #1}}

%\newcommand{\T}{\mathrm{T}}                                % T used in matrix transpose
%\newcommand{\tauarrow}{\stackrel{\tau}{\rightarrow}}       % the symbol tau over a right arrow
%\newcommand{\q}{\bfm{q}}
%\newcommand{\p}{\bfm{p}}
%\newcommand{\z}{\bfm{z}}
%\newcommand{\expect}[1]{\langle #1 \rangle}                % <.> for denoting expectations over realizations of an experiment or thermal averages
%\newcommand{\estimator}[1]{\hat{#1}}                       % estimator for some quantity from a finite dataset.
%%\newcommand{\bfm}[1]{\mbox{\boldmath{$#1$}}}               % proper boldface for math environment
%\newcommand{\bfm}[1]{{\bf #1}}
\newcommand{\program}[1]{\textsc{\lowercase{#1}}}  % Formatting to be used for program or package names.  This uses tiny 


\newcommand{\bra}[1]{\langle #1|}                          % Dirac bra <.|
\newcommand{\ket}[1]{|#1\rangle}                           % Dirac ket |.>
\newcommand{\braket}[2]{\langle #1|#2\rangle}              % Dirac bra-ket <.|.>

\newcommand{\oparg}{\:\cdot\:}           % argument of an operator
\newcommand{\set}[1]{\mathcal{S}_{#1}}           % argument of an operator

