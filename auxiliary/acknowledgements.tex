% Acknowledge:
% - Julie Ransom
% - Bill Swope (de facto thesis advisor)
% - Ken Dill (thesis advisor)
% - Libusha (my ethical and scientific guide)
% - Peggy (for many years of fun)
% - My parents
% - Darby the cat (for putting up with me while doing this whole Ph.D. thing)
% - John Prentice (freelance physicist extraordaire)
% - Terry Lang (for being a scientific partner in crime)
% - The Dill lab (esp. Vince Voelz, Justin Bradford, Banu, and Kings)
% - The Biophysics/BMI/CCB kids (esp. Terry Lang, Mike Kim, Michael Reese)
% - Jerry Solomon and David Liney (for setting me on this path in the first place)
% - Jed Pitera (I seem to be following him around)
% - Vageli (who can actually solve problems)
% - Vijay / Nina
% - John Prentice
% - Peter Kollman
% - Lillian Chong
% - Del Tomeoni

No scientist works in a vacuum.
The scientific output of each individual is shaped to no small degree by those who have touched their lives deeply.
This dissertation is no exception.
I am indebted to so many people for their contributions to this work through direct and indirect means that it is not possible to name them all here, but I will try to touch on a few whose omission would certainly be a great injustice.

I thank the following people, not in order of importance, but rather mostly in order of appearance:
My mother, who taught me to read;
my father, who first showed me that it is possible to understand how the world around us works;
my constant childhood friend and mentor Delfo Tomeoni, who instilled within me the joy of finding things out;
Jr-Gang Cheng, who somehow recognized the seeds of an actual scientist within me while I was still a green-around-the-ears undergraduate, and taught me how to hold a pipetman;
Jerry Solomon and David Liney, who introduced me to protein folding, the problem that would consume much of my waking thoughts for the next ten years, and to the computational tools that could be used to understand it;
Peggy Gabriel, who made all those undergraduate years a lot more fun;
Peter Kollman, whose booming voice echoing through the hallways as he excitedly talked about free energy calculations I can still recall today;
Lillian Chong, my first rotation advisor, for ``showing me the ropes'' with AMBER and molecular dynamics simulation, whom I was delighted to have the chance to work with again at IBM;
Ken Dill, whose gentle guidance has made my graduate career a wonderful experience, and whose clarity of insight helped me develop a much deeper understanding of statistical mechanics;
the Dill lab crew, especially Banu Ozkan, whose work inspired my course in this thesis, Larry Schweitzer for keeping everything running so smoothly, and my classmates, Justin Bradford, Vincent Voelz, and Byoung-Chul Lee;
Bill Swope, whose scientific enthusiasm, attention to detail, and thoroughness I have come to greatly appreciate, and whom, along with Jed Pitera, I had the privilege to work with on body of work this dissertation encompasses;
Libusha Kelly for being my constant scientific and ethical sounding board, someone I can always count on to put me in my place when I screw up, and without whose constant encouragement this dissertation would have never been finished;
Nina Singhal and Vijay Pande, whom it has been my pleasure to work with throughout much of this dissertation;
Matt Jacobson, for always keeping his office door open, which let to numerous enjoyable and enlightening scientific discussions;
the Brass at UCSF kids, especially Alvin Mok, Brian Fife, Ann Wehman, Nick Endres, Terry Lang, and Ian Harwood, for ensuring there was at least \emph{one} night a week I wasn't working;
the Biophysics/BMI/CCB kids, especially Terry Lang, Michael Kim, and Michael Reese, for making science a whole lot of fun;
and finally, but perhaps most importantly, Julie Ransom, the benevolent shepherd of the Biophysics program, without whose tireless efforts I would surely be destitute, lying in a ditch somewhere in Tijuana, clutching a cheap bottle of bourbon. 

\begin{center}
$\bfm{\cdots}$
\end{center}

My thesis committee, consisting of Ken Dill (chair), Matt Jacobson, and Vijay Pande, deserves special thanks for reviewing this dissertation.  
The text of Chapter 2 largely contains material to appear in \emph{Journal of Chemical Theory and Computation}.
Chapter 3 contains material to appear in \emph{Multiscale Modeling and Simulation}.
Chapter 4 contains material to be submitted to the \emph{Journal of Physical Chemistry B}.
In these three chapters, co-authors made the following contributions: Bill Swope aided in the development of the theory and the interpretation of data and helped write the manuscript, and Jed Pitera participated in discussions and provided code upon which the parallel tempering simulation code is based.
Chaok Seok also contributed to development of some of the early theoretical development for Chapter 2, and implemented some early test code.
Ken Dill supervised the research that produced these chapters.
Chapter 5 contains material to be submitted to the \emph{Journal of Chemical Physics}.
There, Nina Singhal and I cowrote the manuscript with the assistance of Bill Swope, and Nina was responsible for the majority of the application of the algorithm to the various test systems and interpretation of the results.
Both Ken Dill and Vijay Pande supervised the research that produced this chapter.
