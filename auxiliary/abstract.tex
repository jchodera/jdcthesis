Abstract goes here.

%The master equation formalism is a powerful tool for describing the dynamics of a system whose dynamics can be characterized by a discrete-state, continuous-time Markov process.  
%The evolution of experimentally measurable dynamical observables can be computed and compared with experiment.  
%Dynamical information not directly accessible to experiment, such as folding pathways, transiently populated conformations, and mean first passage times, can also be easily obtained.  
%We are investigating methods for constructing master equation models of protein dynamics from realistic molecular mechanics models, where metastable subsets of conformation space are represented by discrete states.  
%We believe this approach will provide a useful bridge between the short timescales accessible by molecular dynamics simulation and the long timescales accessible by experiment.

%In order to construct discrete-state, continuous-time master equation models from molecular mechanics models of proteins, two essential components are necessary: 
%(1) a method of dividing up configuration space into metastable subsets, and (2) a way of efficiently computing transition rates between these subsets.  
%In the past year, we have made significant progress on (2).  
%Parallel tempering (or replica-exchange among temperatures) molecular dynamics simulations are used to efficiently explore conformational energy landscapes and generate ensembles of short trajectory segments at several temperatures.  
%We have extended histogram reweighting methods to allow dynamical observables, such as transition probabilities between conformational subsets, to be optimally estimated at any temperature in the range spanned by the replica temperatures.  
%This allows efficient estimation of transitions that are rare at low temperatures.
%Coupled with an alogrithm to partition configuration space, we will be able to construct master equation models at various temperatures, replicating popular experimental nonequilibrium conditions, such as T-jump unfolding.

%My current research focuses on understanding common features of protein folding mechanisms and protein dynamics by the use of atomistic simulations.  
%Though molecular dynamics trajectories can only reach relatively short timescales, through the use of many short trajectories generated by parallel computers, we can build meso-scale models of conformational dynamics through the construction of Markov models representing transitions between metastable conformational substates. 
%Recent results indicate that this approach is able to accurately describe the long-time dynamics of small peptides in explicit solvent.

%In order to develop a detailed understanding of biomolecular dynamics and protein folding kinetics, we need to characterize the statistical dynamics of the molecular system over times that are long compared to the timescales accessible by molecular dynamics simulation. 
%Luckily, due to the hierarchical nature of the energy landscape and the presence of a viscous solvent, a natural separation of timescales makes it possible to model dynamics as a series of infrequent transitions between metastable conformational substates. 
%If phase space is fragmented into a sufficient number of conformational substates, these interstate transitions can be well characterized by large numbers of molecular dynamics simulations in parallel or by importance sampling techniques. 
%The resulting model is a discrete-state, continuous-time Markov process, where we can make use of the master equation formalism for which there exist powerful tools for the study of dynamical behavior. 
%We demonstrate that such a model accurately describes the dynamics of a small solvated peptide system and discuss our recent work in developing efficient multistage algorithms for construction of these models from explicit-solvent molecular dynamics simulations.

